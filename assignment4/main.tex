\documentclass[journal,12pt,onecolumn]{IEEEtran}
\usepackage{cite}
\usepackage{amsmath,amssymb,amsfonts,amsthm}
\usepackage{algorithmic}
\usepackage{graphicx}
\usepackage{textcomp}
\usepackage{xcolor}
\usepackage{txfonts}
\usepackage{listings}
\usepackage{enumitem}
\usepackage{mathtools}
\usepackage{gensymb}
\usepackage{comment}
\usepackage[breaklinks=true]{hyperref}
\usepackage{tkz-euclide} 
\usepackage{listings}
\usepackage{gvv}                                        
\usepackage[latin1]{inputenc}                                
\usepackage{color}                                            
\usepackage{array}                                            
\usepackage{longtable}                                       
\usepackage{calc}                                             
\usepackage{multirow}                                         
\usepackage{hhline}                                           
\usepackage{ifthen}                                           
\usepackage{lscape}
\usepackage{tabularx}
\usepackage{array}
\usepackage{float}
\usepackage{multicol}

\newtheorem{theorem}{Theorem}[section]
\newtheorem{problem}{Problem}
\newtheorem{proposition}{Proposition}[section]
\newtheorem{lemma}{Lemma}[section]
\newtheorem{corollary}[theorem]{Corollary}
\newtheorem{example}{Example}[section]
\newtheorem{definition}[problem]{Definition}
\newcommand{\BEQA}{\begin{eqnarray}}
\newcommand{\EEQA}{\end{eqnarray}}
\newcommand{\define}{\stackrel{\triangle}{=}}
\theoremstyle{remark}
\newtheorem{rem}{Remark}

% Marks the beginning of the document
\begin{document}
\bibliographystyle{IEEEtran}
\vspace{3cm}

\title{Assignment 4}
\author{DESABOINA SRI SATHWIK-AI24BTECH11007}
\maketitle
% Removed \newpage to avoid a blank first page
\bigskip

\renewcommand{\thefigure}{\theenumi}
\renewcommand{\thetable}{\theenumi}
\section*{JEE MAINS-2021(session-2: 5th shift)}
\subsection*{Section-A}
\begin{enumerate}
	\item
		If the functions are defined as $f(x)=\sqrt{x}$ and $g(x)=\sqrt{1-x}$, then what is the common domain of the following functions: $f+g$, $f-g$, $f/g$?

		\hfill{(Mar-2021)}
		\begin{multicols}{4}
		\begin{enumerate}
    \item $0 < x \leq 1$
    \item $0 \leq x < 1$
    \item $0 \leq x \leq 1$
    \item $0 < x < 1$
                \end{enumerate}
		\end{multicols}
	\item
		Let $\alpha, \beta, \gamma$ be the roots of the equation $x^3 + ax^2 + bx + c = 0$ (where $a, b, c \in \mathbb{R}$ and $a \neq 0, b \neq 0$). The system of equations in $u, v, w$ given by $\alpha u + \beta v + \gamma w = 0$, $\beta u + \gamma v + \alpha w = 0$, $\gamma u + \alpha v + \beta w = 0$ has non-trivial solutions. Then the value of $\frac{a^2}{b}$ is:

			\hfill{(Mar-2021)}
		\begin{multicols}{4}
               \begin{enumerate}
    \item 5
    \item 1
    \item 0
    \item 3
               \end{enumerate}
		\end{multicols}
       \item
	       If the equation $a{|Z|}^2 + \bar{\alpha}Z + \alpha \bar{Z} + d = 0$ represents a circle where $a$,$d$ are real constants, then which of the following condition is correct?

			\hfill{(Mar-2021)}
			\begin{multicols}{2}
		\begin{enumerate}
    \item $|\alpha|^2 - ad \neq 0$
    \item $|\alpha|^2 - ad > 0 \quad \text{and} \quad a \in \mathbb{R} \setminus \{0\}$
    \item $\alpha = 0, \quad a, d \in \mathbb{R}^+$
    \item $|\alpha|^2 - ad \geq 0 \quad \text{and} \quad a \in \mathbb{R}$
                \end{enumerate}
			\end{multicols}
	\item
		$ \frac{1}{32 - 1} + \frac{1}{52 - 1} + \frac{1}{72 - 1} + \ldots + \frac{1}{2012 - 1} $ is equal to:

			\hfill{(Mar-2021)}
			\begin{multicols}{4}
                \begin{enumerate}
    \item $\frac{101}{404}$
    \item $\frac{101}{408}$
    \item $\frac{99}{400}$
    \item $\frac{25}{101}$
                \end{enumerate}
			\end{multicols}
	\item
		The number of integral values of $m$ such that the abscissa of the point of intersection of the lines $3x + 4y = 9$ and $y = mx + 1$ is also an integer, is:

			\hfill{(Mar-2021)}
			\begin{multicols}{4}
                \begin{enumerate}
    \item 3
    \item 2
    \item 1
    \item 0
                \end{enumerate}
			\end{multicols}
	\item
		The solutions of the equation $ \det\begin{bmatrix}
1 + \sin^2 x & \sin^2 x & \sin^2 x \\
\cos^2 x & 1 + \cos^2 x & \cos^2 x \\
4\sin(2x) & 4\sin(2x) & 1 + 4\sin(2x)
		\end{bmatrix} = 0 $, $(0 < x < \pi),$ are:

			\hfill{(Mar-2021)}
			\begin{multicols}{2}
		\begin{enumerate}
    \item $ \frac{\pi}{6}, \frac{5\pi}{6} $
    \item $ \frac{7\pi}{12}, \frac{11\pi}{12} $
    \item $ \frac{5\pi}{12}, \frac{7\pi}{12} $
    \item $ \frac{\pi}{12}, \frac{\pi}{6} $
                \end{enumerate}
			\end{multicols}
	\item
			If $f(x) = \begin{cases} \frac{1}{|x|} & \text{if } x \geq 1 \\ ax^2 + b & \text{if } |x| < 1 \end{cases}$ is differentiable at every point of the domain, then the values of $a$ and $b$ are respectively:

					\hfill{(Mar-2021)}
					\begin{multicols}{4}
                \begin{enumerate}
    \item $ \frac{5}{2}, -\frac{3}{2} $
    \item $ -\frac{1}{2}, \frac{3}{2} $
    \item $ \frac{1}{2}, \frac{1}{2} $
    \item $ \frac{1}{2}, -\frac{3}{2} $
                \end{enumerate}
					\end{multicols}
	\item
		A vector $a$ has components $3p$ and $1$ with respect to a rectangular Cartesian system. This system is rotated through a certain angle about the origin in the counterclockwise sense. If with respect to the new system, $a$ has components $p+1$ and $\sqrt{10}$, then a value of $p$ is equal to:

			\hfill{(Mar-2021)}
			\begin{multicols}{4}
		\begin{enumerate}
    \item $ 1 $
    \item $ -1 $
    \item $ \frac{4}{5} $
    \item $ -\frac{5}{4} $
                \end{enumerate}
			\end{multicols}
	\item
		The sum of all the 4-digit distinct numbers that can be formed with the digits 1,2,2 and 3 is:

			\hfill{(Mar-2021)}
			\begin{multicols}{2}
		\begin{enumerate}
    \item $ 26664 $
    \item $ 122664 $
    \item $ 122234 $
    \item $ 22264 $
                \end{enumerate}
			\end{multicols}
	\item
                Choose the correct statement about two circles whose equations are given below:\\
$ x^2 + y^2 - 10x - 10y + 41 = 0 $\\
$ x^2 + y^2 - 22x - 10y + 137 = 0 $\\

	\hfill{(Mar-2021)}
	\begin{multicols}{2}
                \begin{enumerate}
    \item circles have no meeting point
    \item circles have two meeting points
    \item circles have only one meeting point
    \item circles have the same centre
                \end{enumerate}
	\end{multicols}
	\item
                If $\alpha$, $\beta$ are natural numbers such that $100^{\alpha} - 199\beta = (100)(100) + (99)(101) + (98)(102) + \ldots + (1)(199)$, then the slope of the line passing through $(\alpha ,\beta)$ and the origin is:

			\hfill{(Mar-2021)}
			\begin{multicols}{4}
                \begin{enumerate}
    \item $ 510 $
    \item $ 550 $
    \item $ 540 $
    \item $ 530 $
                \end{enumerate}
			\end{multicols}
	\item
		The value of \\
		$ 3 + \frac{1}{4 + \frac{1}{3 + \frac{1}{4 + \frac{1}{3 + \ldots \infty}}}} $  \\
		is equal to:

			\hfill{(Mar-2021)}
			\begin{multicols}{2}
		\begin{enumerate}
    \item $ 3 + 2\sqrt{3} $
    \item $ 4 + \sqrt{3} $
    \item $ 2 + \sqrt{3} $
    \item $ 1.5 + \sqrt{3} $
                \end{enumerate}
			\end{multicols}
	\item
		The integral $
\int \frac{(2x - 1) \cos\left(\sqrt{(2x - 1)^2 + 5}\right)}{\sqrt{4x^2 - 4x + 6}} \, dx $ is equal to (where \( c \) is a constant of integration):

	\hfill{(Mar-2021)}
	\begin{multicols}{2}
               \begin{enumerate}
    \item $ \frac{1}{2} \sin\left(\sqrt{(2x + 1)^2 + 5}\right) + c $
    \item $ \frac{1}{2} \sin\left(\sqrt{(2x - 1)^2 + 5}\right) + c $
    \item $ \frac{1}{2} \cos\left(\sqrt{(2x + 1)^2 + 5}\right) + c $
    \item $ \frac{1}{2} \cos\left(\sqrt{(2x - 1)^2 + 5}\right) + c $
               \end{enumerate}
	\end{multicols}
       \item
	       The differential equations satisfied by the system of parabolas $ y^2 = 4a(x + a)$ is:

	       	\hfill{(Mar-2021)}
		\begin{multicols}{2}
	       \begin{enumerate}
    \item $ y \frac{dy}{dx} + 2x \frac{dy}{dx} - y = 0 $
    \item $ y \left(\frac{dy}{dx}\right)^2 + 2x \frac{dy}{dx} - y = 0 $
    \item $ y \left(\frac{dy}{dx}\right)^2 - 2x \frac{dy}{dx} - y = 0 $
    \item $ y \left(\frac{dy}{dx}\right)^2 - 2x \frac{dy}{dx} + y = 0 $
               \end{enumerate}
		\end{multicols}
       \item
	       The real-valued function $ f(x) = \frac{\cosec^{-1}(x)}{\sqrt{x - [x]}}$ where $[x]$ denotes the greatest integer less than or equal to $x$, is defined for all $x$ belonging to:

	       	\hfill{(Mar-2021)}
		\begin{multicols}{1}
	       \begin{enumerate}
    \item all non-integers except the interval $[-1, 1]$
    \item all integers except $0$, $-1$, $1$
    \item all reals except integers
    \item all reals except the interval $[-1, 1]$
               \end{enumerate}
		\end{multicols}














\end{enumerate}

\end{document}
