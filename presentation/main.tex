\documentclass{beamer}
\mode<presentation>
\usepackage{amsmath}
\usepackage{amssymb}
%\usepackage{advdate}
\usepackage{adjustbox}
\usepackage{subcaption}
\usepackage{enumitem}
\usepackage{multicol}
\usepackage{mathtools}
\usepackage{listings}
\usepackage{url}
\def\UrlBreaks{\do\/\do-}
\usetheme{Boadilla}
\usecolortheme{lily}
\setbeamertemplate{footline}
{
  \leavevmode%
  \hbox{%
  \begin{beamercolorbox}[wd=\paperwidth,ht=2.25ex,dp=1ex,right]{author in head/foot}%
    \insertframenumber{} / \inserttotalframenumber\hspace*{2ex} 
  \end{beamercolorbox}}%
  \vskip0pt%
}
\setbeamertemplate{navigation symbols}{}

\providecommand{\nCr}[2]{\,^{#1}C_{#2}} % nCr
\providecommand{\nPr}[2]{\,^{#1}P_{#2}} % nPr
\providecommand{\mbf}{\mathbf}
\providecommand{\pr}[1]{\ensuremath{\Pr\left(#1\right)}}
\providecommand{\qfunc}[1]{\ensuremath{Q\left(#1\right)}}
\providecommand{\sbrak}[1]{\ensuremath{{}\left[#1\right]}}
\providecommand{\lsbrak}[1]{\ensuremath{{}\left[#1\right.}}
\providecommand{\rsbrak}[1]{\ensuremath{{}\left.#1\right]}}
\providecommand{\brak}[1]{\ensuremath{\left(#1\right)}}
\providecommand{\lbrak}[1]{\ensuremath{\left(#1\right.}}
\providecommand{\rbrak}[1]{\ensuremath{\left.#1\right)}}
\providecommand{\cbrak}[1]{\ensuremath{\left\{#1\right\}}}
\providecommand{\lcbrak}[1]{\ensuremath{\left\{#1\right.}}
\providecommand{\rcbrak}[1]{\ensuremath{\left.#1\right\}}}
\theoremstyle{remark}
\newtheorem{rem}{Remark}
\newcommand{\sgn}{\mathop{\mathrm{sgn}}}
\providecommand{\abs}[1]{\left\vert#1\right\vert}
\providecommand{\res}[1]{\Res\displaylimits_{#1}} 
\providecommand{\norm}[1]{\lVert#1\rVert}
\providecommand{\mtx}[1]{\mathbf{#1}}
\providecommand{\mean}[1]{E\left[ #1 \right]}
\providecommand{\fourier}{\overset{\mathcal{F}}{ \rightleftharpoons}}
%\providecommand{\hilbert}{\overset{\mathcal{H}}{ \rightleftharpoons}}
\providecommand{\system}{\overset{\mathcal{H}}{ \longleftrightarrow}}
	%\newcommand{\solution}[2]{\textbf{Solution:}{#1}}
%\newcommand{\solution}{\noindent \textbf{Solution: }}
\providecommand{\dec}[2]{\ensuremath{\overset{#1}{\underset{#2}{\gtrless}}}}
\newcommand{\myvec}[1]{\ensuremath{\begin{pmatrix}#1\end{pmatrix}}}
\let\vec\mathbf

\lstset{
%language=C,
frame=single, 
breaklines=true,
columns=fullflexible
}

\numberwithin{equation}{section}

\title{Presentation Template}
\author{Sri Sathwik Desaboina \\ AI24BTECH11007}

\date{\today} 
\begin{document}

\begin{frame}
\titlepage
\end{frame}

\section*{Outline}
\begin{frame}
\tableofcontents
\end{frame}
\section{Problem}
\begin{frame}
\frametitle{Problem Statement}
%

Show that the point $\myvec{x \\ y}$ given by $x = \frac{2at}{1 + t^2}$ and $y = \frac{a(1 - t^2)}{1 + t^2}$ lies on a circle for all real values of $t$ such that $-1 \leq t \leq 1$, where $a$ is any given real number.
\end{frame}

%\subsection{Literature}
\section{Solution}
\subsection{Usage of variables}
\begin{frame}
\frametitle{Usage of variables}
	\begin{tabular}[12pt]{ |c| c| c|}
    \hline
    \textbf{Point} & \textbf{Description}&\textbf{Coordinates}\\ 
    \hline
	$\vec O$ & centre of the circle&$\myvec{4\\4}$\\
    \hline
	$\vec A$ & A Point on the circle & $\myvec{4\\0}$\\
    \hline
	$\vec B$ & Diametrically opposite point to A & ? \\ 
    \hline
    \end{tabular}

\end{frame}
\subsection{Parametric form}
\begin{frame}
\frametitle{Parametric form}
Given $x$ and $y$ in the parametric form,\\
	\begin{align}
		x &= \frac{2at}{1 + t^2},\\
		\quad y &= \frac{a(1 - t^2)}{1 + t^2}
	\end{align}
	Let $\vec{p(t)}$ be equal to,\\
	\begin{align}
	\mathbf{p}(t) = \myvec{x \\ y} = \myvec{\frac{2at}{1 + t^2} \\ \frac{a(1 - t^2)}{1 + t^2}}.
	\end{align}


\end{frame}
\subsection{Verification}
\begin{frame}
\frametitle{Verification}
	The transformation matrix $\vec{A(t)}$ with parameter $t$ is,\\
	\begin{align}
		\mathbf{A}(t) & = \myvec{ \frac{2a}{1+t^2} & 0 \\ 0 & \frac{a(1-t^2)}{1+t^2}},\\ \text{Then, we get} \vec{p(t)},\\ \mathbf{p}(t) & = \myvec{ \frac{2a}{1+t^2} & 0 \\ 0 & \frac{a(1 - t^2)}{1+t^2} } \myvec{t \\ 1},\\ \text{That implies, we get}\\ \mathbf{p}(t)^T \myvec{ 1 & 0 \\ 0 & 1} \mathbf{p}(t) = a^2\\ \text{Therefore, we can say that}\\
	x^2 + y^2 = a^2.
	\end{align}
\end{frame}
%\section{Plot}
\subsection{Balanced Equation}
\begin{frame}[fragile]
\frametitle{Balanced Equation}

Thus, 
\begin{align}
\label{eq:chem_balance_mat_sol}
x_1 &= \frac{3}{4}x_4, x_2 = x_4, x_3 = \frac{1}{4}x_4
\\
\implies \vec{x} &= x_4\myvec{\frac{3}{4} \\ 1 \\ \frac{1}{4} \\ 1}= \myvec{3 \\ 4 \\ 1 \\ 4}
\end{align}
%
upon substituting $x_4 = 4$.
%
\eqref{eq:chem_balance_unsol} then becomes
%
\begin{align}
\label{eq:chem_balance_final}
3Fe+4H_2 O &\rightarrow Fe_3 O_4 + 4H_2
\end{align}
The codes in 
{\footnotesize
\begin{lstlisting}
https://github.com/DESABOINASRISATHWIK/EE1030/blob/main/presentation/codes/plot.py
https://github.com/DESABOINASRISATHWIK/EE1030/blob/main/presentation/codes/code.c
\end{lstlisting}
}
verifies \eqref{eq:chem_balance_mat_sol}.
%plots Fig. \ref{fig:circle_diameter}.
%
%\begin{figure}
%\centering
%\includegraphics[width=0.6\columnwidth]{./figs/circle_diameter.eps}
%\caption{Circle $C$ and all lines (i)-(iv). (ii) is a diameter.}
%\label{fig:circle_diameter}
%\end{figure}
\end{frame}
%\begin{frame}
%\frametitle{Introduction}
%\framesubtitle{Literature}
%%\begin{figure}[t!]
%%    \centering
%%    \begin{subfigure}[t]{0.4\columnwidth}
%%        \centering
%%        \includegraphics[width=\columnwidth]{point_source}
%%        \caption{Single point source}
%%\label{fig3:subfig1}        
%%    \end{subfigure}%
%%    ~ 
%%    \begin{subfigure}[t]{0.4\columnwidth}
%%        \centering
%%        \includegraphics[width=\columnwidth]{pointNoPowerDist_new}
%%        \caption{SNR profile}
%%\label{fig3:subfig2}
%%    \end{subfigure}
%%  %  \caption{Average SNR for a BPP. $N=16$}
%%    \label{fig3}
%%  \end{figure}
%
%\end{frame}
%  
%
%
%%

\end{document}
